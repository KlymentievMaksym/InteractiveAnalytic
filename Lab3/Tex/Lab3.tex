\documentclass{article}

\input{../../packages.tex}

\graphicspath{ {../Images/} }
\begin{document}
    \begin{titlepage}
        \begin{center}

        Міністерство освіти і науки України
        
        НТУУ «Київський політехнічний інститут»
        
        Фізико-технічний інститут
        \vspace{3.3cm}
        
        {\textbf{Системи та засоби інтерактивної аналітики}\\Лабораторна робота No3\\Прості SQL запити\\Варіант No6}

        \vspace{8.5cm}

        \begin{flushright}
            \textbf{Виконав:}\\Студент 4-го курсу\\групи ФІ-21\\Климентьєв Максим\\
            \textbf{Перевірив:}\\\text{\_\_\_\_\_\_\_\_\_\_\_\_\_\_\_\_\_\_}
        \end{flushright}

        \end{center}
    \end{titlepage}
    \newpage

    \pagenumbering{gobble}
    \tableofcontents
    \cleardoublepage
    \pagenumbering{arabic}
    \setcounter{page}{3}

    \newpage
    \section{Мета роботи}
        Навчитися створювати прості SQL запити. 

    \newpage
    \section{Завдання}
    \begin{enumerate}
        \item Перевірити на правильність БД, створену у лабораторній роботі №2.
        \item Вивчити запити SELECT, INSERT, DELETE, UPDATE, CREATE, DROP, ALTER.
        \item Зробити запити наведені у Табл. 3.1.

        \item Створити звіт. Приєднати до класу. \begin{itemize}
            \item У звіті навести все необхідне для повторення і перевірки ваших дій (діаграму БД, SQL запити для створення БД і таблиць, структуру БД, і т.д). Навести знімки екрана, які підтверджують виконані дії.
            \item У протоколі SQL запити наводити у текстовому вигляді щоб їх можна було редагувати і модифікувати під час захисту.
            \item Зробити висновки по роботі і занести їх у звіт.
        \end{itemize}

        \item Підготувати відповіді на контрольні питання (для офлайн захисту навести їх у протоколі, розкрити сутність, навести приклади).

        \item Захистити роботу.

    \end{enumerate}

    \begin{table}[!h]
        \centering
        \begin{tabular}{| c | m{32em} |} 
            \hline
            № варіанта & Завдання \\ [0.5ex] 
            \hline
            6 & 
                \underline{\textbf{Проект БД (лаб. роб. 2)}} \newline
                \underline{Магазин рослин, облік} \newline
                \textbf{№; Назва; Склад; Постачальник; Особливості; Відповідальні} \newline
                1; Фікус Каріка; №1; UGT, AVDtrade; Садові, Закритого грунту; Луцик М. В., Ступак Я. К. \newline
                2; Аденіум огрядний; №2; AVDtrade; Кімнатні;Падик В. О.  \newline
                3; Аглаонема; №1, №3; КвітиУкраїни; Кімнатні; Ступак Я. К. \newline
                … \newline
                341; Азалія; №1; Волошка; Садові; Луцик М. В. \newline \newline
                \textbf{Запити:} \newline
                1. До таблиці рослини слід додати поле типу DATE з назвою RegDate. \newline
                2. Додати нову рослину \newline
                3. Змінити Фікус Каріка на Фікус Бенджамина \newline
                4. Вибрати  всіх рослини з інформацією про них \newline
                5. Вибрати всі рослини, які надійшли на склад пізніше певної дати (дату вибрати самостійно, поле RegDate) \\
            \hline
        \end{tabular}
    \end{table}

    \newpage
    \section{Код реалізації}

    \begin{lstlisting}[language=sql]
SELECT * FROM tbl_plant; 
    \end{lstlisting}

    \begin{figure}[!h]
        \includegraphics[width=1\linewidth]{select1.png}
    \end{figure}

    \newpage
    \begin{lstlisting}[language=sql]
ALTER TABLE tbl_plant
  ADD COLUMN RegDate DATE;

INSERT INTO tbl_plant
            (name,
             RegDate)
VALUES      ('Монстера Деліціоза',
             '2025-10-01');

UPDATE tbl_plant
SET    name = 'Фікус Бенджамина'
WHERE  name = 'Фікус Каріка';

SELECT *
FROM   tbl_plant; 
    \end{lstlisting}

    \begin{figure}[!h]
        \includegraphics[width=0.8\linewidth]{select2.png}
    \end{figure}

    \newpage
    \begin{lstlisting}[language=sql]
UPDATE tbl_plant
SET    RegDate = Date_add('2025-09-01', INTERVAL Floor(Rand() * 40) day)
WHERE  RegDate IS NULL;

SELECT *
FROM   tbl_plant
WHERE  RegDate > '2025-09-20'; 
    \end{lstlisting}

    \begin{figure}[!h]
        \includegraphics[width=1\linewidth]{select3.png}
    \end{figure}

    \newpage
    \section{Висновки}
        У ході лабораторної роботи було створено та виконано прості SQL-запити для роботи з базою даних "Магазин рослин". Закріплено навички додавання, зміни, вибірки й оновлення даних за допомогою основних команд SQL.

    \newpage
    \section{Контрольні питання}

    \begin{enumerate}
        \item \textbf{Що таке SQL запит?}\\
            SQL запит --- це вираз, написаний мовою SQL, що дає змогу виконувати операції з даними в базі даних, такі як вибірка, вставка, оновлення та видалення даних.
        \item \textbf{Яким запитом можна змінити структуру створеної раніше таблиці?}\\
            \texttt{ALTER TABLE}.\\
            Приклад: \texttt{ALTER TABLE tbl\_plant ADD COLUMN RegDate DATE;}
        \item \textbf{Яким запитом додають нові записи у таблицю?}\\
            \texttt{INSERT INTO}.\\
            Приклад: \texttt{INSERT INTO tbl\_plant (name, RegDate) VALUES ('Монстера Деліціоза', '2025-10-01');}
        \item \textbf{Яким запитом видаляють записи з таблиці?}\\
            \texttt{DELETE}.\\
            Приклад: \texttt{DELETE FROM tbl\_plant WHERE id = 3;}
        \item \textbf{Яким запитом оновлюють значення у таблиці?}\\
            \texttt{UPDATE}.\\
            Приклад: \texttt{UPDATE tbl\_plant SET name = 'Фікус Бенджамина' WHERE name = 'Фікус Каріка';}
        \item \textbf{Яким запитом вибирають значення з таблиці?}\\
            \texttt{SELECT}.\\
            Приклад: \texttt{SELECT * FROM tbl\_plant;}
        \item \textbf{Яким вибрати тільки ті записи, у яких поле price має значення менше числа 43?}\\
            \texttt{WHERE}.\\
            Приклад: \texttt{SELECT * FROM products WHERE price < 43;}
    \end{enumerate}


\end{document}