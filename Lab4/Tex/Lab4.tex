\documentclass{article}

\input{../../packages.tex}

\graphicspath{ {../Images/} }
\begin{document}
    \begin{titlepage}
        \begin{center}

        Міністерство освіти і науки України
        
        НТУУ «Київський політехнічний інститут»
        
        Фізико-технічний інститут
        \vspace{3.3cm}
        
        {\textbf{Системи та засоби інтерактивної аналітики}\\Лабораторна робота No4\\Запити розширеного SQL (advanced SQL). RegExp\\Варіант No6}

        \vspace{8.5cm}

        \begin{flushright}
            \textbf{Виконав:}\\Студент 4-го курсу\\групи ФІ-21\\Климентьєв Максим\\
            \textbf{Перевірив:}\\\text{\_\_\_\_\_\_\_\_\_\_\_\_\_\_\_\_\_\_}
        \end{flushright}

        \end{center}
    \end{titlepage}
    \newpage

    \pagenumbering{gobble}
    \tableofcontents
    \cleardoublepage
    \pagenumbering{arabic}
    \setcounter{page}{3}

    \newpage
    \section{Мета роботи}
        Навчитися створювати розширені SQL запити і опанувати роботу з RegExp.

    \newpage
    \section{Завдання}
    Ґакери намагаються зламати ваш сайт і зіпсувати бузу даних.
    Треба знайти записи, з нестандартними діями, ідентифікувати IP адреси комп’ютерів з яких відбувається атака і передати ці записи до кіберполіції.
    Можна створювати додаткові таблиці.\\
    \colorbox{LightCoral}{\textbf{Запити повинні фільтрувати інформацію точно, а не близько до завдання.}}
    
    \begin{enumerate}
        \item До БД створеної у попередніх роботах імпортувати таблицю з даними.

        \item Зробити запити наведені у Табл. 4.1.\\
            \colorbox{LightCoral}{Обов’язково використати regexp.}\\
            \colorbox{LightGoldenrod}{Зверніть увагу, кодів завершення багато.}

        \item Створити звіт. Приєднати до класу. \begin{itemize}
            \item У звіті навести все необхідне для повторення і перевірки ваших дій (діаграму БД, SQL запити для створення БД і таблиць, структуру БД, і т.д). Навести знімки екрана, які підтверджують виконані дії.
            \item У протоколі SQL запити наводити у текстовому вигляді щоб їх можна було редагувати і модифікувати під час захисту.
            \item Зробити висновки по роботі і занести їх у звіт.
        \end{itemize}

        \item Підготувати відповіді на контрольні питання (для офлайн захисту навести їх у протоколі, розкрити сутність, навести приклади).

        \item Захистити роботу.

    \end{enumerate}

    \begin{table}[!h]
        \centering
        \begin{tabular}{| c | m{32em} |} 
            \hline
            № варіанта & Завдання \\ [0.5ex] 
            \hline
            6 & 
                \underline{Використати існуючу БД (лаб. роб. 2,3)}\newline
                \textbf{Запити:}\newline
                1. Вивести повні адреси JS скриптів (розширення js) і їх розмір, які запросили з ком’пютера з IP адресою 83.227.29.211, для яких запит завершився вдало.\newline
                2. Придумайте будь який запит з використанням UNION, в дослідити різні форми цієї команди.\newline
                3. Знайдіть розмір всього скачаного комп'ютером з IP адресою  83.227.29.211\\
            \hline
        \end{tabular}
    \end{table}

    \newpage
    \section{Код реалізації}

    \begin{lstlisting}[language=sql]
    \end{lstlisting}

    % \begin{figure}[!h]
    %     \includegraphics[width=1\linewidth]{create.png}
    % \end{figure}

    % \begin{figure}[!h]
    %     \includegraphics[width=1\linewidth]{relation.png}
    % \end{figure}

    \newpage
    \begin{lstlisting}[language=sql]
    \end{lstlisting}

    % \begin{figure}[!h]
    %     \includegraphics[width=1\linewidth]{insert.png}
    % \end{figure}

    \newpage
    \begin{lstlisting}[language=sql]
    \end{lstlisting}

    % \begin{figure}[!h]
    %     \includegraphics[width=1\linewidth]{select.png}
    % \end{figure}

    \newpage
    \section{Висновки}
        Знайдено записи, з нестандартними діями, ідентифікувано IP адреси комп’ютерів з яких відбулась атака і передано ці записи до кіберполіції.

    \newpage
    \section{Контрольні питання}

    \begin{enumerate}
        \item \textbf{Що таке запити розширеного SQL (advanced SQL)?}
        \item \textbf{Що таке регулярні вирази?}
        \item \textbf{Для чого використовуються запити з командою UNION?}
        \item \textbf{Що таке агрегатні функції?}
        \item \textbf{Для чого використовуються запити з командою ORDER?}
        \item \textbf{Для чого використовуються запити з командою HAVING?}
    \end{enumerate}


\end{document}