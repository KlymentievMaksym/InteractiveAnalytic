\documentclass{article}

\input{../../packages.tex}

\graphicspath{ {../Images/} }
\begin{document}
    \begin{titlepage}
        \begin{center}

        Міністерство освіти і науки України
        
        НТУУ «Київський політехнічний інститут»
        
        Фізико-технічний інститут
        \vspace{3.3cm}
        
        {\textbf{Системи та засоби інтерактивної аналітики}\\Лабораторна робота No4\\Запити розширеного SQL (advanced SQL). RegExp\\Варіант No6}

        \vspace{8.5cm}

        \begin{flushright}
            \textbf{Виконав:}\\Студент 4-го курсу\\групи ФІ-21\\Климентьєв Максим\\
            \textbf{Перевірив:}\\\text{\_\_\_\_\_\_\_\_\_\_\_\_\_\_\_\_\_\_}
        \end{flushright}

        \end{center}
    \end{titlepage}
    \newpage

    \pagenumbering{gobble}
    \tableofcontents
    \cleardoublepage
    \pagenumbering{arabic}
    \setcounter{page}{3}

    \newpage
    \section{Мета роботи}
        Навчитися створювати розширені SQL запити і опанувати роботу з RegExp.

    \newpage
    \section{Завдання}

    \begin{figure}[!h]
        \includegraphics[width=1\linewidth]{LogFormat.jpg}
    \end{figure}

    Ґакери намагаються зламати ваш сайт і зіпсувати бузу даних.
    Треба знайти записи, з нестандартними діями, ідентифікувати IP адреси комп’ютерів з яких відбувається атака і передати ці записи до кіберполіції.
    Можна створювати додаткові таблиці.\\
    \colorbox{LightCoral}{\textbf{Запити повинні фільтрувати інформацію точно, а не близько до завдання.}}
    
    \begin{enumerate}
        \item До БД створеної у попередніх роботах імпортувати таблицю з даними.

        \item Зробити запити наведені у Табл. 4.1.\\
            \colorbox{LightCoral}{Обов’язково використати regexp.}\\
            \colorbox{LightGoldenrod}{Зверніть увагу, кодів завершення багато.}

        \item Створити звіт. Приєднати до класу. \begin{itemize}
            \item У звіті навести все необхідне для повторення і перевірки ваших дій (діаграму БД, SQL запити для створення БД і таблиць, структуру БД, і т.д). Навести знімки екрана, які підтверджують виконані дії.
            \item У протоколі SQL запити наводити у текстовому вигляді щоб їх можна було редагувати і модифікувати під час захисту.
            \item Зробити висновки по роботі і занести їх у звіт.
        \end{itemize}

        \item Підготувати відповіді на контрольні питання (для офлайн захисту навести їх у протоколі, розкрити сутність, навести приклади).

        \item Захистити роботу.

    \end{enumerate}

    \begin{table}[!h]
        \centering
        \begin{tabular}{| c | m{32em} |} 
            \hline
            № варіанта & Завдання \\ [0.5ex] 
            \hline
            6 & 
                \underline{Використати існуючу БД (лаб. роб. 2,3)}\newline
                \textbf{Запити:}\newline
                1. Вивести повні адреси JS скриптів (розширення js) і їх розмір, які запросили з ком’пютера з IP адресою 83.227.29.211, для яких запит завершився вдало.\newline
                2. Придумайте будь який запит з використанням UNION, в дослідити різні форми цієї команди.\newline
                3. Знайдіть розмір всього скачаного комп'ютером з IP адресою  83.227.29.211\\
            \hline
        \end{tabular}
    \end{table}

    \newpage
    \section{Код реалізації}

    \begin{figure}[!h]
        \includegraphics[width=1\linewidth]{Import.png}
    \end{figure}

    \begin{figure}[!h]
        \includegraphics[width=1\linewidth]{ImportSuccess.png}
    \end{figure}

    \newpage
    1. Вивести повні адреси JS скриптів (розширення js) і їх розмір, які запросили з ком’пютера з IP адресою 83.227.29.211, для яких запит завершився вдало.
    \begin{lstlisting}[language=sql]
SELECT
	Script,
    Size
FROM (SELECT
        SUBSTRING_INDEX(Line, ' ', 1) as IP,
        SUBSTRING_INDEX(SUBSTRING_INDEX(SUBSTRING_INDEX(SUBSTRING_INDEX(Line, '"', 2), '"', -1), ' ', 2), ' ', -1) as Script,
        SUBSTRING_INDEX(SUBSTRING_INDEX(Line, ' "', 2), ' ', -1) as Size,
        SUBSTRING_INDEX(SUBSTRING_INDEX(SUBSTRING_INDEX(Line, ' "', 2), ' ', -2), ' ', 1) as Status
    FROM `tblaccesslog`) as first_task
WHERE first_task.IP = '83.227.29.211'
	AND first_task.Status REGEXP '^2'
    AND first_task.Script REGEXP '\\.js(\\?.*)?$';
    \end{lstlisting}

    \begin{figure}[!h]
        \includegraphics[width=1\linewidth]{First.png}
    \end{figure}


    \newpage
    2. Придумайте будь який запит з використанням UNION, в дослідити різні форми цієї команди.
    \begin{lstlisting}[language=sql]
(SELECT
	IP,
    Size
FROM (SELECT
        SUBSTRING_INDEX(Line, ' ', 1) as IP,
        SUBSTRING_INDEX(SUBSTRING_INDEX(Line, ' "', 2), ' ', -1) as Size,
      	SUBSTRING_INDEX(SUBSTRING_INDEX(SUBSTRING_INDEX(Line, ' "', 2), ' ', -2), ' ', 1) as Status
    FROM `tblaccesslog`) as second_task
WHERE second_task.IP REGEXP '173.255.176.5|83.169.39.166'
	AND second_task.Size = 0
    AND second_task.Status REGEXP '^2')
UNION
(SELECT
	IP,
    Size
FROM (SELECT
        SUBSTRING_INDEX(Line, ' ', 1) as IP,
        SUBSTRING_INDEX(SUBSTRING_INDEX(Line, ' "', 2), ' ', -1) as Size,
      	SUBSTRING_INDEX(SUBSTRING_INDEX(SUBSTRING_INDEX(Line, ' "', 2), ' ', -2), ' ', 1) as Status
    FROM `tblaccesslog`) as second_task
WHERE second_task.IP REGEXP '83.227.29.211'
	AND second_task.Size < 200
    AND second_task.Status REGEXP '^2');
    \end{lstlisting}

    \begin{figure}[!h]
        \includegraphics[width=0.25\linewidth]{Second1.png}
    \end{figure}

    \newpage
    \begin{lstlisting}[language=sql]
(SELECT
	IP,
    Size
FROM (SELECT
        SUBSTRING_INDEX(Line, ' ', 1) as IP,
        SUBSTRING_INDEX(SUBSTRING_INDEX(Line, ' "', 2), ' ', -1) as Size,
      	SUBSTRING_INDEX(SUBSTRING_INDEX(SUBSTRING_INDEX(Line, ' "', 2), ' ', -2), ' ', 1) as Status
    FROM `tblaccesslog`) as second_task
WHERE second_task.IP REGEXP '173.255.176.5|83.169.39.166'
	AND second_task.Size = 0
    AND second_task.Status REGEXP '^2')
UNION ALL
(SELECT
	IP,
    Size
FROM (SELECT
        SUBSTRING_INDEX(Line, ' ', 1) as IP,
        SUBSTRING_INDEX(SUBSTRING_INDEX(Line, ' "', 2), ' ', -1) as Size,
      	SUBSTRING_INDEX(SUBSTRING_INDEX(SUBSTRING_INDEX(Line, ' "', 2), ' ', -2), ' ', 1) as Status
    FROM `tblaccesslog`) as second_task
WHERE second_task.IP REGEXP '83.227.29.211'
	AND second_task.Size < 200
    AND second_task.Status REGEXP '^2');
    \end{lstlisting}

    \begin{figure}[!h]
        \includegraphics[width=0.25\linewidth]{Second2.png}
    \end{figure}

    \newpage
    3. Знайдіть розмір всього скачаного комп'ютером з IP адресою  83.227.29.211
    \begin{lstlisting}[language=sql]
SELECT
	IP,
    sum(Size) as Total
FROM (SELECT
        SUBSTRING_INDEX(Line, ' ', 1) as IP,
        SUBSTRING_INDEX(SUBSTRING_INDEX(Line, ' "', 2), ' ', -1) as Size
    FROM `tblaccesslog`) as third_task
WHERE third_task.IP = '83.227.29.211'
GROUP BY third_task.IP;
    \end{lstlisting}

    \begin{figure}[!h]
        \includegraphics[width=0.5\linewidth]{Third.png}
    \end{figure}

    \newpage
    \section{Висновки}
        Набуто навички роботи з RegExp у SQL-запитах.

    \newpage
    \section{Контрольні питання}

    \begin{enumerate}
        \item \textbf{Що таке запити розширеного SQL (advanced SQL)?} \\
            Розширені SQL-запити — це запити, які використовують підзапити, об’єднання, агрегації та умови.

        \item \textbf{Що таке регулярні вирази?} \\
            Регулярні вирази --- це спеціальна мова шаблонів для пошуку або перевірки текстових даних. \\
            Приклад: 
            \begin{lstlisting}[language=sql]
SELECT * FROM logs WHERE Line REGEXP '\\.js$';
            \end{lstlisting}
            Знайде всі рядки, що закінчуються на \texttt{.js}.

        \item \textbf{Для чого використовуються запити з командою UNION?} \\
            Команда \texttt{UNION} об'єднання результатів двох селектів виключаючи повторні рядки, тоді як \texttt{UNION ALL} залишає всі записи. 
            \begin{lstlisting}[language=sql]
SELECT name FROM students
UNION
SELECT name FROM teachers;
            \end{lstlisting}
            Поверне список усіх імен без повторів.

        \item \textbf{Що таке агрегатні функції?} \\
            Агрегатні функції --- це функція, які повертають одинарне значення з колекції вхідних значень такої як множина.
            До них належать:
            \begin{itemize}
                \item \texttt{COUNT()} --- кількість рядків;
                \item \texttt{SUM()} --- сума значень;
                \item \texttt{AVG()} --- середнє значення;
                \item \texttt{MIN()}, \texttt{MAX()} --- мінімальне та максимальне значення.
            \end{itemize}

        \item \textbf{Для чого використовуються запити з командою ORDER?} \\
            Команда \texttt{ORDER BY} використовується для впорядкування (сортування) набору результатів у порядку зростання або спадання.\\
            Наприклад, якщо потрібно продивитись юзерів за спаданням їх віку.
            \begin{lstlisting}[language=sql]
SELECT name, age FROM users ORDER BY age DESC;
            \end{lstlisting}

        \item \textbf{Для чого використовуються запити з командою HAVING?} \\
            Команда \texttt{HAVING} використовується якщо треба накласти умову на результат агрегатної функції.
            \begin{lstlisting}[language=sql]
SELECT department, COUNT(*) AS workers
FROM employees
GROUP BY department
HAVING workers > 10;
            \end{lstlisting}
            Повертає лише ті відділи, де кількість працівників більша за 10.
    \end{enumerate}


\end{document}