\documentclass{article}

\input{../../packages.tex}

\graphicspath{ {../Images/} }
\begin{document}
    \begin{titlepage}
        \begin{center}

        Міністерство освіти і науки України
        
        НТУУ «Київський політехнічний інститут»
        
        Фізико-технічний інститут
        \vspace{3.3cm}
        
        {\textbf{Системи та засоби інтерактивної аналітики}\\Лабораторна робота No2\\Нормальні форми\\Варіант No6}

        \vspace{8.5cm}

        \begin{flushright}
            \textbf{Виконав:}\\Студент 4-го курсу\\групи ФІ-21\\Климентьєв Максим\\
            \textbf{Перевірив:}\\\text{\_\_\_\_\_\_\_\_\_\_\_\_\_\_\_\_\_\_}
        \end{flushright}

        \end{center}
    \end{titlepage}
    \newpage

    \pagenumbering{gobble}
    \tableofcontents
    \cleardoublepage
    \pagenumbering{arabic}
    \setcounter{page}{3}

    \newpage
    \section{Мета роботи}
        Ознайомитися з процесом нормалізації баз даних і привести базу даних до третьої нормальної форми.

    \newpage
    \section{Завдання}
    \begin{enumerate}
        \item Створити базу даних, назву вибрати самостійно.

        \item Варіанти у Табл 2.1 \begin{itemize}
            \item Задати для всіх таблиць типи полів.
            \item Структуру таблиць вибрати керуючись логікою і правилами реляційних БД у 3й номальній формі .
            \item Заповнити довільними значеннями, схожими на правду. Кожна таблиця 3..5 записів.
            \item Варіанти за нашим списком студентів, 11 варіант = 1. 12 = 2 далі, аналогічно.
            \item Крапка з комою (;) розділяє поля, кома (,) розділяє декілька значень у одному полі.
        \end{itemize}

        \item Створити звіт. Приєднати до класу. \begin{itemize}
            \item У звіті навести все необхідне для повторення і перевірки ваших дій (діаграму БД, SQL запити для створення БД і таблиць, структуру БД, і т.д). Навести знімки екрана, які підтверджують виконані дії.
            \item У протоколі SQL запити наводити у текстовому вигляді щоб їх можна було редагувати і модифікувати під час захисту.
            \item Зробити висновки по роботі і занести їх у звіт.
        \end{itemize}

        \item Підготувати відповіді на контрольні питання (для офлайн захисту навести їх у протоколі, розкрити сутність, навести приклади).

        \item Захистити роботу.

    \end{enumerate}

    \begin{table}[!h]
        \centering
        \begin{tabular}{| c | m{32em} |} 
            \hline
            № варіанта & Завдання \\ [0.5ex] 
            \hline
            6 & 
                \underline{Магазин рослин, облік} \newline
                \textbf{№; Назва; Склад; Постачальник; Особливості; Відповідальні} \newline
                1; Фікус Каріка; №1; UGT, AVDtrade; Садові, Закритого грунту; Луцик М. В., Ступак Я. К. \newline
                2; Аденіум огрядний; №2; AVDtrade; Кімнатні;Падик В. О.  \newline
                3; Аглаонема; №1, №3; КвітиУкраїни; Кімнатні; Ступак Я. К. \newline
                … \newline
                341; Азалія; №1; Волошка; Садові; Луцик М. В.\\
            \hline
        \end{tabular}
    \end{table}

    \newpage
    \section{Код реалізації}

    \begin{lstlisting}[language=sql]
CREATE TABLE tbl_plant
  (
     id   INT PRIMARY KEY auto_increment,
     name VARCHAR(100) NOT NULL
  );

CREATE TABLE tbl_feature
  (
     id      INT PRIMARY KEY auto_increment,
     feature VARCHAR(100) NOT NULL
  );

CREATE TABLE tbl_plant_feature
  (
     plant_id   INT NOT NULL,
     feature_id INT NOT NULL,
     FOREIGN KEY (plant_id) REFERENCES tbl_plant(id),
     FOREIGN KEY (feature_id) REFERENCES tbl_feature(id),
     UNIQUE (plant_id, feature_id)
  );

CREATE TABLE tbl_storage
  (
     id   INT PRIMARY KEY auto_increment,
     name VARCHAR(100) NOT NULL
  );

CREATE TABLE tbl_responsible
  (
     id   INT PRIMARY KEY auto_increment,
     name VARCHAR(100) NOT NULL
  );

CREATE TABLE tbl_storage_responsible
  (
     id             INT PRIMARY KEY auto_increment,
     storage_id     INT NOT NULL,
     responsible_id INT NOT NULL,
     FOREIGN KEY (storage_id) REFERENCES tbl_storage(id),
     FOREIGN KEY (responsible_id) REFERENCES tbl_responsible(id),
     UNIQUE (storage_id, responsible_id)
  );

CREATE TABLE tbl_supplier
  (
     id   INT PRIMARY KEY auto_increment,
     name VARCHAR(100) NOT NULL
  );

CREATE TABLE tbl_accounting
  (
     plant_id               INT NOT NULL,
     storage_responsible_id INT NOT NULL,
     supplier_id            INT NOT NULL,
     FOREIGN KEY (plant_id) REFERENCES tbl_plant(id),
     FOREIGN KEY (supplier_id) REFERENCES tbl_supplier(id),
     FOREIGN KEY (storage_responsible_id) REFERENCES tbl_storage_responsible(id),
     UNIQUE (plant_id, storage_responsible_id, supplier_id)
  ); 
    \end{lstlisting}

    \begin{figure}[!h]
        \includegraphics[width=1\linewidth]{create.png}
    \end{figure}

    \begin{figure}[!h]
        \includegraphics[width=1\linewidth]{relation.png}
    \end{figure}

    \newpage
    \begin{lstlisting}[language=sql]
INSERT INTO tbl_plant
            (NAME)
VALUES      ('Фікус Каріка'),
            ('Аденіум огрядний'),
            ('Аглаонема'),
            ('Азалія'),
            ('Сансевієрія'),
            ('Орхідея Фаленопсис'),
            ('Пеларгонія'),
            ('Монстера'),
            ('Бегонія'),
            ('Каланхоє');

INSERT INTO tbl_feature
            (feature)
VALUES      ('Садові'),
            ('Закритого ґрунту'),
            ('Кімнатні'),
            ('Суцвіття'),
            ('Листяні'),
            ('Тропічні'),
            ('Флора кімнатна'),
            ('Вологолюбні'),
            ('Ароматні'),
            ('Квітучі');

INSERT INTO tbl_plant_feature
            (plant_id,
             feature_id)
VALUES      (1,
             1),
            (1,
             2), -- Фікус Каріка
            (2,
             3), -- Аденіум огрядний
            (3,
             3), -- Аглаонема
            (4,
             1), -- Азалія
            (5,
             3),
            (5,
             5), -- Сансевієрія
            (6,
             3),
            (6,
             10), -- Орхідея Фаленопсис
            (7,
             3),
            (7,
             9), -- Пеларгонія
            (8,
             6),
            (8,
             5), -- Монстера
            (9,
             3),
            (9,
             10), -- Бегонія
            (10,
             3),
            (10,
             8); -- Каланхоє
INSERT INTO tbl_storage
            (NAME)
VALUES      ('№1'),
            ('№2'),
            ('№3'),
            ('№4'),
            ('№5'),
            ('№6'),
            ('№7'),
            ('№8'),
            ('№9'),
            ('№10');

INSERT INTO tbl_responsible
            (NAME)
VALUES      ('Луцик М. В.'),
            ('Ступак Я. К.'),
            ('Падик В. О.'),
            ('Іваненко О. П.'),
            ('Ковальчук С. А.'),
            ('Мельник Т. В.'),
            ('Бондаренко Л. М.'),
            ('Сидоренко Н. І.'),
            ('Гончарук В. Ю.'),
            ('Романюк Ю. В.');

INSERT INTO tbl_storage_responsible
            (storage_id,
             responsible_id)
VALUES      (1,
             1), -- №1 → Луцик
            (1,
             2), -- №1 → Ступак
            (2,
             3), -- №2 → Падик
            (3,
             2), -- №3 → Ступак
            (4,
             4), -- №4 → Іваненко
            (5,
             5), -- №5 → Ковальчук
            (6,
             6), -- №6 → Мельник
            (7,
             7), -- №7 → Бондаренко
            (8,
             8), -- №8 → Сидоренко
            (9,
             9), -- №9 → Гончарук
            (10,
             10); -- №10 → Романюк
INSERT INTO tbl_supplier
            (NAME)
VALUES      ('UGT'),
            ('AVDtrade'),
            ('КвітиУкраїни'),
            ('Волошка'),
            ('GreenHouse'),
            ('FloraMix'),
            ('BotanicShop'),
            ('PlantWorld'),
            ('EcoGarden'),
            ('TropiFlowers');

INSERT INTO tbl_accounting
            (plant_id,
             storage_responsible_id,
             supplier_id)
VALUES      (1,
             1,
             1), -- Фікус Каріка, №1→Луцик, UGT
            (1,
             2,
             2), -- Фікус Каріка, №1→Ступак, AVDtrade
            (1,
             4,
             2), -- Фікус Каріка, №3→Ступак, AVDtrade
            (2,
             3,
             2), -- Аденіум, №2→Падик, AVDtrade
            (3,
             4,
             3), -- Аглаонема, №3→Ступак, КвітиУкраїни
            (4,
             5,
             4), -- Азалія, №4→Іваненко, Волошка
            (5,
             6,
             5), -- Сансевієрія, №6→Мельник, GreenHouse
            (6,
             7,
             6), -- Орхідея, №7→Бондаренко, FloraMix
            (7,
             8,
             7), -- Пеларгонія, №8→Сидоренко, BotanicShop
            (8,
             9,
             8), -- Монстера, №9→Гончарук, PlantWorld
            (9,
             10,
             9); -- Бегонія, №10→Романюк, EcoGarden
    \end{lstlisting}

    \begin{figure}[!h]
        \includegraphics[width=1\linewidth]{insert.png}
    \end{figure}

    \newpage
    \begin{lstlisting}[language=sql]
SELECT p.name                                          AS Рослина,
       Group_concat(DISTINCT f.feature SEPARATOR ', ') AS Особливості,
       Group_concat(DISTINCT s.name SEPARATOR ', ')    AS Постачальники,
       Group_concat(DISTINCT st.name SEPARATOR ', ')   AS Склади,
       Group_concat(DISTINCT r.name SEPARATOR ', ')    AS Відповідальні
FROM   tbl_plant AS p
       INNER JOIN tbl_accounting AS a
               ON p.id = a.plant_id
       INNER JOIN tbl_supplier AS s
               ON a.supplier_id = s.id
       INNER JOIN tbl_storage_responsible AS sr
               ON a.storage_responsible_id = sr.id
       INNER JOIN tbl_storage AS st
               ON sr.storage_id = st.id
       INNER JOIN tbl_responsible AS r
               ON sr.responsible_id = r.id
       LEFT JOIN tbl_plant_feature AS pf
              ON p.id = pf.plant_id
       LEFT JOIN tbl_feature AS f
              ON pf.feature_id = f.id
GROUP  BY p.name
ORDER  BY p.name;  
    \end{lstlisting}

    \begin{figure}[!h]
        \includegraphics[width=1\linewidth]{select.png}
    \end{figure}

    \newpage
    \section{Висновки}
        Створено структуру бази даних "Магазин рослин", нормалізовану до 3НФ. Виділено основні сутності, встановлено зв’язки між ними та усунуто надлишковість даних.

    \newpage
    \section{Контрольні питання}

    \begin{enumerate}
        \item \textbf{Що таке нормальні форми?} \\
        Нормальні форми — правила, які застосовують до БД, за допомогою яких відбувається оптимізація структури БД, що дозволяє зробити БД оптимальнішою, ефективнішою. 

        \item \textbf{Опишіть вимоги першої нормальної форми (1НФ).}
        \begin{itemize}
            \item Атомарність.
            \item Уникнення повторень груп правильно визначаючи неключові атрибути.
            \item Основний ключ: мінімальний набір колонок, які ідентифікують запис.
        \end{itemize}
        

        \item \textbf{Опишіть вимоги другої нормальної форми (2НФ).}
        \begin{itemize}
            \item 1НФ.
            \item Дані, що зберігаються в таблицях з композитним ключем, не залежать лише від частини ключа.
            \item Дані, що повторно з'являються в декількох рядках, виносяться в окремі таблиці.
        \end{itemize}

        \item \textbf{Опишіть вимоги третьої нормальної форми (3НФ).}
        \begin{itemize}
            \item 2НФ.
            \item Дані в таблиці залежали винятково від основного ключа.
            \item Без транзитивних залежностей (немає залежності неключових атрибутів один від одного).
        \end{itemize}

        \item \textbf{Продемонструйте на простому прикладі приведення БД до 3НФ.} \\
        \textit{Приклад:}\\
        Початкова таблиця:\\
        \begin{tabular}{|c|c|c|}
            \hline
            Студент & Група & Кафедра \\
            \hline
            Іванов & ФІ-21 & ММАД\\
            \hline
            Петров & ФІ-21 & ММАД\\
            \hline
            Сидоренко & ФІ-23 & ММЗІ\\
            \hline
        \end{tabular}
        \begin{itemize}
            \item Тут є транзитивна залежність: \texttt{Група → Кафедра}.
            \item Для 3НФ виділяємо окремі таблиці:  
                \begin{itemize}
                    \item \textbf{Студенти}(id, Ім'я, Група\_id)  
                    \item \textbf{Групи}(id, Назва, Кафедра)  
                \end{itemize}
            \item Тепер кожен неключовий атрибут залежить лише від первинного ключа.
        \end{itemize}
    \end{enumerate}


\end{document}