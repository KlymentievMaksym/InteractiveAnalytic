\documentclass{article}

\input{../../packages.tex}

\graphicspath{ {../Images/} }

\begin{document}
    \begin{titlepage}
        \begin{center}

        Міністерство освіти і науки України
        
        НТУУ «Київський політехнічний інститут»
        
        Фізико-технічний інститут
        \vspace{3.3cm}
        
        {\textbf{Системи та засоби інтерактивної аналітики}\\Комп’ютерний практикум No1\\Створення БД в середовищі MySQL\\Варіант No10}

        \vspace{10cm}

        \begin{flushright}
            \textbf{Виконав:}\\Студент 4-го курсу\\групи ФІ-21\\Климентьєв Максим\\
            \textbf{Перевірив:}\\\text{\_\_\_\_\_\_\_\_\_\_\_\_\_\_\_\_\_\_}
        \end{flushright}

        \end{center}
    \end{titlepage}
    \newpage

    \pagenumbering{gobble}
    \tableofcontents
    \cleardoublepage
    \pagenumbering{arabic}
    \setcounter{page}{3}

    \newpage
    \section{Мета роботи}
        Навчитися створювати бази даних у середовищі MySQL.

    \newpage
    \section{Завдання}
    \begin{enumerate}
        \item Створити базу даних, назву вибрати самостійно.

        \item Створити 4 таблиці, назви вибрати самостійно. Варіанти у Табл 1.1 \begin{itemize}
            \item Задати для всіх таблиць типи полів.
            \item Структуру таблиці вибрати керуючись логікою завдання, навести її у вигляді діаграми.
            \item Наповнити довільними значеннями, схожими на правду. Кожна таблиця 3..5 записів.
        \end{itemize}

        \item Створити звіт. Приєднати до класу. \begin{itemize}
            \item У звіті навести все необхідне для повторення і перевірки ваших дій (SQL запити для створення БД і таблиць, структуру БД, і т.д). Навести знімки екрана, які підтверджують виконані дії.
            \item У протоколі SQL запити наводити у текстовому вигляді щоб їх можна було редагувати і модифікувати під час захисту.
            \item Зробити висновки по роботі і занести їх у звіт.
        \end{itemize}

        \item Підготувати відповіді на контрольні питання (для офлайн захисту навести їх у протоколі, розкрити сутність, навести приклади).

        \item Захистити роботу.

    \end{enumerate}

    \begin{table}[!h]
        \centering
        \begin{tabular}{| c | m{32em} |} 
            \hline
            № варіанта & Завдання \\ [0.5ex] 
            \hline
            10 & Створити базу даних на тему "Магазин іграшок". Створити такі таблиці. "Назва іграшки", "Постачальники", "Тип іграшки", "Вартість". Побудувати всі необхідні зв'язки.\\
            \hline
        \end{tabular}
    \end{table}

    \newpage
    \section{Код реалізації}

    % \begin{figure}[!h]
    %     \includegraphics[width=0.8\linewidth]{Scheme.png}
    % \end{figure}

    \begin{figure}[!h]
        \includegraphics[width=0.8\linewidth]{Scheme_v2.png}
    \end{figure}

    \begin{lstlisting}[language=SQL]
CREATE TABLE tbl_toy_name
  (
     id   INT PRIMARY KEY auto_increment,
     name VARCHAR(256) NOT NULL
  );

CREATE TABLE tbl_suppliers
  (
     id       INT PRIMARY KEY auto_increment,
     name     VARCHAR(256) NOT NULL,
     contacts TEXT
  );

CREATE TABLE tbl_toy_type
  (
     id       INT PRIMARY KEY auto_increment,
     color    VARCHAR(256) NOT NULL,
     material VARCHAR(256) NOT NULL
  );

CREATE TABLE tbl_price
  (
     toy_name_id INT NOT NULL,
     toy_type_id INT NOT NULL,
     supplier_id INT NOT NULL,
     price       FLOAT,
     FOREIGN KEY (toy_name_id) REFERENCES tbl_toy_name(id),
     FOREIGN KEY (toy_type_id) REFERENCES tbl_toy_type(id),
     FOREIGN KEY (supplier_id) REFERENCES tbl_suppliers(id)
  );
    \end{lstlisting}

    \begin{figure}[!h]
        \includegraphics[width=1\linewidth]{Create.png}
    \end{figure}

    \begin{lstlisting}[language=SQL]
INSERT INTO tbl_toy_name
            (NAME)
VALUES      ("bear"),
            ("bee"),
            ("rabbit"),
            ("mouse"),
            ("cat");

INSERT INTO tbl_toy_type
            (color,
             material)
VALUES      ("green",
             "cotton"),
            ("red",
             "wood"),
            ("white",
             "glass"),
            ("green",
             "glass"),
            ("red",
             "cotton");

INSERT INTO tbl_suppliers
            (NAME,
             contacts)
VALUES      ("gregory",
             "vault street, 13, 89"),
            ("hyperion",
             "+380998884433"),
            ("atlas",
             "+380991211122"),
            ("cov",
             "+380991198301"),
            ("rozetka",
             "+380990003322");  
    \end{lstlisting}

    \begin{figure}[!h]
        \includegraphics[width=1\linewidth]{Insert.png}
    \end{figure}

    \begin{lstlisting}[language=SQL]
INSERT INTO tbl_price
VALUES      (1,
             1,
             2,
             100.20 ),
            (1,
             2,
             2,
             150.50 ),
            (3,
             3,
             1,
             999.99 ),
            (4,
             5,
             4,
             220.20 ),
            (5,
             1,
             3,
             230.30 ),
            (5,
             3,
             5,
             250.10 );
    \end{lstlisting}

    \begin{figure}[!h]
        \includegraphics[width=1\linewidth]{InsertP.png}
    \end{figure}

    \begin{lstlisting}[language=SQL]
        SELECT * FROM tbl_price;
    \end{lstlisting}

    \begin{figure}[!h]
        \includegraphics[width=1\linewidth]{SelectPrice.png}
    \end{figure}

    \begin{lstlisting}[language=SQL]
        SELECT * FROM tbl_suppliers;
    \end{lstlisting}

    \begin{figure}[!h]
        \includegraphics[width=1\linewidth]{SelectSuppliers.png}
    \end{figure}

    \newpage

    \begin{lstlisting}[language=SQL]
        SELECT * FROM tbl_toy_name;
    \end{lstlisting}

    \begin{figure}[!h]
        \includegraphics[width=1\linewidth]{SelectName.png}
    \end{figure}

    \begin{lstlisting}[language=SQL]
        SELECT * FROM tbl_toy_type;
    \end{lstlisting}

    \begin{figure}[!h]
        \includegraphics[width=1\linewidth]{SelectType.png}
    \end{figure}

    \newpage
    \section{Висновки}
        У ході лабораторної роботи було спроєктовано та створено базу даних "Магазин іграшок". Було розроблено структуру та побудовано всі необхідні зв’язки між таблицями.

    \newpage
    % \section{Контрольні питання}

    % \begin{enumerate}
    %     \item Що таке бази даних? \eqref{fig:database}
    %     \item Чим характерні реляційні БД? \eqref{fig:database_rel}
    %     \item Що таке поле? \eqref{fig:field}
    %     \item Що таке запис? \eqref{fig:record}
    %     \item Які обмеження на параметрі сучасних реляційних БД (розмір, кількість записів, …)? \eqref{fig:database_limits}
    %     \item Базові команди SQL? \eqref{fig:database_commands}
    % \end{enumerate}

    % \begin{enumerate}
    %     \item \begin{figure}[h!]\label{fig:database}
    %         Сховище впорядкованих даних під керуванням СУБД.
    %     \end{figure}

    %     \item \begin{figure}[h!]\label{fig:database_rel}
    %         Дані у таблицях, зв’язки через ключі, робота SQL.
    %     \end{figure}

    %     \item \begin{figure}[h!]\label{fig:field}
    %         Стовпець таблиці.
    %     \end{figure}

    %     \item \begin{figure}[h!]\label{fig:record}
    %         Рядок таблиці.
    %     \end{figure}

    %     \item \begin{figure}[h!]\label{fig:database_limits}
    %         Таблиця може мати до ~4 млрд рядків, розмір БД — терабайти (MySQL).
    %     \end{figure}

    %     \item \begin{figure}[h!]\label{fig:database_commands}
    %         CREATE, DROP, ALTER, SELECT, INSERT, UPDATE, DELETE
    %     \end{figure}
    % \end{enumerate}
\end{document}