\documentclass{article}

\input{../../packages.tex}

\graphicspath{ {../Images/} }

\begin{document}
    \begin{titlepage}
        \begin{center}

        Міністерство освіти і науки України
        
        НТУУ «Київський політехнічний інститут»
        
        Фізико-технічний інститут
        \vspace{3.3cm}
        
        {\textbf{Системи та засоби інтерактивної аналітики}\\Комп’ютерний практикум No1\\Створення БД в середовищі MySQL\\Варіант No10}

        \vspace{10cm}

        \begin{flushright}
            \textbf{Виконав:}\\Студент 4-го курсу\\групи ФІ-21\\Климентьєв Максим\\
            \textbf{Перевірив:}\\\text{\_\_\_\_\_\_\_\_\_\_\_\_\_\_\_\_\_\_}
        \end{flushright}

        \end{center}
    \end{titlepage}
    \newpage

    \pagenumbering{gobble}
    \tableofcontents
    \cleardoublepage
    \pagenumbering{arabic}
    \setcounter{page}{3}

    \newpage
    \section{Мета роботи}
        Навчитися створювати базі даних у середовищі MySQL.

    \newpage
    \section{Завдання}
    \begin{enumerate}
        \item Створити базу даних, назву вибрати самостійно.

        \item Створити 4 таблиці, назви вибрати самостійно. Варіанти у Табл 1.1 \begin{itemize}
            \item Задати для всіх таблиць типи полів.
            \item Структуру таблиці вибрати керуючись логікою завдання, навести її у вигляді діаграми.
            \item Наповнити довільними значеннями, схожими на правду. Кожна таблиця 3..5 записів.
        \end{itemize}

        \item Створити звіт. Приєднати до класу. \begin{itemize}
            \item У звіті навести все необхідне для повторення і перевірки ваших дій (SQL запити для створення БД і таблиць, структуру БД, і т.д). Навести знімки екрана, які підтверджують виконані дії.
            \item У протоколі SQL запити наводити у текстовому вигляді щоб їх можна було редагувати і модифікувати під час захисту.
            \item Зробити висновки по роботі і занести їх у звіт.
        \end{itemize}

        \item Підготувати відповіді на контрольні питання (для офлайн захисту навести їх у протоколі, розкрити сутність, навести приклади).

        \item Захистити роботу.

    \end{enumerate}

    \begin{table}[!h]
        \centering
        \begin{tabular}{| c | m{32em} |} 
            \hline
            № варіанта & Завдання \\ [0.5ex] 
            \hline
            10 & Створити базу даних на тему "Магазин іграшок". Створити такі таблиці. "Назва іграшки", "Постачальники", "Тип іграшки", "Вартість". Побудувати всі необхідні зв'язки.\\
            \hline
        \end{tabular}
    \end{table}

    \newpage
    \section{Код реалізації}

    \begin{figure}[!h]
        \includegraphics[width=0.8\linewidth]{Scheme.png}
    \end{figure}

    \newpage
    \section{Висновки}

    \newpage
    \section{Контрольні питання}

    \begin{enumerate}
        \item Що таке бази даних? \eqref{fig:database}
        \item Чим характерні реляційні БД? \eqref{fig:database_rel}
        \item Що таке поле? \eqref{fig:field}
        \item Що таке запис? \eqref{fig:record}
        \item Які обмеження на параметрі сучасних реляційних БД (розмір, кількість записів, …)? \eqref{fig:database_limits}
        \item Базові команди SQL? \eqref{fig:database_commands}
    \end{enumerate}

    \begin{enumerate}
        \item \begin{figure}[h!]\label{fig:database}
            
        \end{figure}

        \item \begin{figure}[h!]\label{fig:database_rel}
            
        \end{figure}

        \item \begin{figure}[h!]\label{fig:field}
            
        \end{figure}

        \item \begin{figure}[h!]\label{fig:record}
            
        \end{figure}

        \item \begin{figure}[h!]\label{fig:database_limits}
            
        \end{figure}

        \item \begin{figure}[h!]\label{fig:database_commands}
            
        \end{figure}
    \end{enumerate}
\end{document}