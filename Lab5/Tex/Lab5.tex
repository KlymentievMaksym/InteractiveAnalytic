\documentclass{article}

\input{../../packages.tex}

\graphicspath{ {../Images/} }
\begin{document}
    \begin{titlepage}
        \begin{center}

        Міністерство освіти і науки України
        
        НТУУ «Київський політехнічний інститут»
        
        Фізико-технічний інститут
        \vspace{3.3cm}
        
        {\textbf{Системи та засоби інтерактивної аналітики}\\Лабораторна робота No5\\Автоматизація збирання і обробки інформації\\Варіант No6}

        \vspace{8.5cm}

        \begin{flushright}
            \textbf{Виконав:}\\Студент 4-го курсу\\групи ФІ-21\\Климентьєв Максим\\
            \textbf{Перевірив:}\\\text{\_\_\_\_\_\_\_\_\_\_\_\_\_\_\_\_\_\_}
        \end{flushright}

        \end{center}
    \end{titlepage}
    \newpage

    \pagenumbering{gobble}
    \tableofcontents
    \cleardoublepage
    \pagenumbering{arabic}
    \setcounter{page}{3}

    \newpage
    \section{Мета роботи}
        Навчитися автоматизовано отримувати інформацію з сторінки сайта, попередньо її обробляти і записувати у базу даних для подальшої роботи.

    \newpage
    \section{Завдання}
    Основна задача - навчитися обробляти інформацію \underline{автоматизовано}.
    \textbf{Всі дії} зробити за допомогою програми на мові програмування, яку ви знаєте.
    
    \begin{enumerate}
        \item \underline{\textbf{Імпорт інформації.}}\\
            Файл завантажити у БД, якщо Ваш комп'ютер повільний, можна з файла взяти меньший об'єм інформації.\\
            \textbf{На 14 або 15 балів} файл імпортувати у базу як є, кожен рядок -> один запис у базі (ID, TEXT), потім SQL запитом з регекспами перетворит таблицю на таблицю з записами де кожне поле логфайла відповідає полю у таблиці (ID, IP, DATA, URL, RETCODE, SIZE, ….).
            Потім створити SQL запит, яким порахувати вказане у п2. І програмно виміряти час, за який виконаються пп 2-3.
            Потім зробити аналогічні дії без використання БД, програмно виміряти час, за який виконаються пп 1-2 і порівняти ресурсоємність двох підходів.\\
            \textbf{На 0-13 балів} зробити пункт 2 будь-яким способом, без поріаняння ресурсоємності.

        \item \underline{\textbf{Аналіз даних.}}\\
            Знайти розміри скачаних даних за кожним кодом стану (всі, не тільки 2хх), які були скачані з певної IP адреси, адреси для кожного варіанту наведені у Табл. 5.1.
        \item \underline{\textbf{Візуалізація.}}\\
            Вивести діаграму розподілу скачаного для перших 3х кодів (з попереднього пункта) і зберегти її у файл (програмно). 
        \item Вивести всю можливу інформацію про IP адресу (країна, місто, провайдер і т.д.).
        \item \underline{Постаратися все зробити у одній програмі. IP - задати змінною для того щоб можна було оперативно формувати запити для інших адрес.}

        \item Створити звіт. Приєднати до класу. \begin{itemize}
            \item У звіті навести все необхідне для повторення і перевірки ваших дій (діаграму БД, SQL запити для створення БД і таблиць, структуру БД, і т.д). Навести знімки екрана, які підтверджують виконані дії.
            \item У протоколі SQL запити наводити у текстовому вигляді щоб їх можна було редагувати і модифікувати під час захисту.
            \item Зробити висновки по роботі і занести їх у звіт.
        \end{itemize}

        \item Підготувати відповіді на контрольні питання (для офлайн захисту навести їх у протоколі, розкрити сутність, навести приклади).

        \item Захистити роботу.

    \end{enumerate}

    Можна використовувати будь яку мову програмування, “картинку” треба згенерувати “статично”, без використання online бібліотек на зразок Chart.js, Google Charts або Chartist, але можна їх використати додатково.


    \begin{table}[!h]
        \centering
        \begin{tabular}{| c | m{32em} |} 
            \hline
            № варіанта & Завдання \\ [0.5ex] 
            \hline
            6 & \colorbox{LightGreen}{46.125.249.79}\\
            \hline
        \end{tabular}
    \end{table}

    \newpage
    \section{Код реалізації}

    \begin{lstlisting}[language=sql]
    \end{lstlisting}

    % \begin{figure}[!h]
    %     \includegraphics[width=1\linewidth]{create.png}
    % \end{figure}

    % \begin{figure}[!h]
    %     \includegraphics[width=1\linewidth]{relation.png}
    % \end{figure}

    \newpage
    \begin{lstlisting}[language=sql]
    \end{lstlisting}

    % \begin{figure}[!h]
    %     \includegraphics[width=1\linewidth]{insert.png}
    % \end{figure}

    \newpage
    \begin{lstlisting}[language=sql]
    \end{lstlisting}

    % \begin{figure}[!h]
    %     \includegraphics[width=1\linewidth]{select.png}
    % \end{figure}

    \newpage
    \section{Висновки}
        Створено програму для автоматизованого імпорту, обробки та аналізу log-файлів із подальшим збереженням у базі даних. Реалізовано побудову SQL-запитів для вибірки даних за IP-адресою, підрахунку обсягів завантажень за кодами стану та візуалізації результатів у вигляді діаграми.

    \newpage
    \section{Контрольні питання}

    \begin{enumerate}
        \item \textbf{Як можна отримувати дані з бази даних за допомогою SQL-запиту і відображати їх на сторінці за допомогою PHP та GD, або у тих мовах і засобах, які ви використали?}
    \end{enumerate}


\end{document}